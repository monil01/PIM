In this paper, we perform a first-order evaluation of HPC kernels developed by laboratories of US Department of Energy (DoE) for Processing in Memory (PIM) technique. To that end, we characterize seven HPC applications and identify memory-intensive kernels which we run on host core and PIM core with HMC main memory module using DAMOV-SIM simulation infrastructure. The results show that two out of the seven HPC kernels achieves a performance gain executing in the PIM core, which is largely analogous to the L3 misses per kilo instructions (MPKI) of each kernel. The results also suggests that MPKI is not the only factor that determines if a kernel would benefit from executing on a PIM core. We observe that a kernel may perform relatively better on a PIM core than another kernel with a higher MPKI but has a lower LFMR. 