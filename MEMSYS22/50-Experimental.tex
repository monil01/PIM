
%In this section we describe the simulator infrastructure target hardware platform, , and benchmarks used to carry out the experiments.

In this section we present the simulation infrastructure and methodology used in the study as well as the metrics that are used to represent the results.  

For our experiments we use recently released DAMOV-SIM simulation infrastructure~\cite{71}. DAMOV is based on ZSim system simulator~\cite{zsim} coupled with Ramulator memory simulator\cite{ramulator}. ZSim is used simulate both host and pim core's microarchitecture, cache hirerchy and coherence protocols. We use Hybrid Memory Cube (HMC)~\cite{hmc} to leverage its logic layer where PIM execution unit is realized. Ramulator memory simulator is used to model HMC main memory architecture, memory controller and main memory access protocols. The system configurations used in the simulations are summarized in Table~\ref{table:hostarch}. DAMOV-SIM uses pintool v. 2.14, on a Linux kernel v.3.8.  

First, we identify the memory-intensive kernels of the benchmarks under study, as detailed in Section 3. We use DAMOV-SIM's offloading support to isolate identified kernels by inserting delimiter functions and run them for 1 billion instructions. In the configuration file, when \texttt{pimMode} is set to \texttt{true} Zsim simulates a pim core with shorter latency to memory with high bandwidth. When \texttt{pimMode} is set to \texttt{false} it simulate a core with regular settings (host processor). At function-level offloading granularity it is easier to determine and comapre how the same function performs in PIM and host cores. %The simulation infrastructure does not yet support concurrent execution on PIM and host core.         

We analyze three important performance metrics: Instruction Per Cycle (IPC), Misses-Per-Kilo-Instruction (MPKI) and Last to First Miss Ratio (LFMR), which indicates the ratio of last level cache misses to first level cache misses.    




%DAMOV-SIM supports function offloading using Zsim delimiters.     




% \subsection{DAMOV}
% - Host CPU Configuration \\
% - PIM CPU configuration \\
% \subsection{HMC}
% - Configuration \\


% \begin{savenotes}
%\begin{table}[t!]
%%\vspace{-0.5cm}
%%\normalsize
%\small
%\caption{DRAM and STT-MRAM parameters associated with row operation (DDR2-667 cycles)}
%%\begin{center}
%    
%    \begin{tabularx}{\columnwidth}{@{}lXrrrr}
%\toprule
%    % \hline
%    Timing \\Parameters & Description & \tiny{DRAM} & \tiny{ST-1.2} & \tiny{ST-1.5} & \tiny{ST-2.0}\\
%\midrule
% tRCD & \mbox{Row to column} \mbox{command delay} & 5 & 6 & 8 & 10\\
% tRP & Row precharge & 5 & 6 & 8 & 10\\
% tFAW & Four row \mbox{activation window} & 13 & 16 & 20 & 26\\
% tRRD & Row to Row \mbox{activation delay} & 3 & 4 & 5 & 6\\
% tRFC & \mbox{Refresh cycle time} & 43 & 0 & 0 & 0\\
%\bottomrule
%   \end{tabularx}
%
%\label{table:ROW}
%\end{table}
%   \end{savenotes}
%
%
%
%
%% 
%
\begin{table}[t!]
%\normalsize
\small
\caption{Configurations used in the simulation }
%\begin{cente
    \begin{tabularx}{\columnwidth}{lXr}
\toprule
    % \hline
Unit & Configuration  \\
\midrule
Host CPU & 1 out-of-order Core running at 2.4GHz,   \\
L1 Cache & 32KB, 8-way, 64B/Line, LRU policy \\
L2 Cache & 256KB, 8-way, 64B/Line, LRU Policy \\
L3 Cache & 8MB, 16-way, 64B/Line, LRU Policy  \\
\midrule
PIM Execution unit & 1 out-of-order Core running at 2.4GHz,   \\
L1 Cache & 32KB, 8-way, 64B/Line, LRU policy \\
\midrule
Main Memory & HMC v.2.0, 32 Vaults, 8 DRAM banks/vault, 256B row buffer, 8GB capacity.\\


\bottomrule
  \end{tabularx}
\label{table:hostarch}
%\vspace{-4mm}
\end{table}
%
%
%
%\begin{figure}[t!]
%\centering
%\includegraphics[width=\columnwidth]{figure/pWCET.png}
%\caption{
%%\textbf{X-axis:} pWCET: Probabilistic worst case execution time (pWCET); 
%%\textbf{Y-axis:} Exceedance probability; 
%%\textbf{Caption:} 
%pWCET distribution: Probability (Y-axis) that the application execution
%time (in any given run) exceeds the corresponding time on the X-axis.
%	In this example, the pWCET estimate (7ms) is exceeded with a probability of $10^{-12}$.}
%\label{fig:pWCET}
%%\vspace{-5mm}
%\end{figure}
%
%
%\begin{figure}[t!]
%\centering
%\includegraphics[width=\columnwidth]{figure/MBPTA-CV.png}
%\caption{
%Schematic of the MBPTA application process}
%\label{fig:MBPTA-CV}
%%\vspace{-5mm}
%\end{figure}
%
%
%
%
%
%\subsection{Benchmarks}
%
%\begin{table}[t!]
%%\normalsize
%\small
%\caption{Benchmarks used in the study}
%%\begin{center}
%    \begin{tabularx}{\columnwidth}{lXrrrr}
%\toprule
%    % \hline
%    Suite & Benchmarks & Domain \\
%    
%\midrule
%ESA Applications & obdp, debie & Space\\
%\midrule
%EEMBC Autobench & a2time01, aifftr01, aifirf01, aiifft01, basefp01, bitmnp01, cacheb01, canrdr01, idctrn01, iirflt01, matrix01, pntrch01, puwmod01, rspeed01, tblook01, ttsprk01 & Automotive\\
%\midrule
%Mediabench & mesa.texgen, mesa.mipmap, epic.decode, mesa.osdemo & Media \\
%\bottomrule
%   \end{tabularx}
%\label{table:benchmark}
%\end{table}
%  


