Memory sub-systems contribute significantly to HPC performance and energy efficiency. Traditional memory technologies with conventional organization (e.g. DRAM) are struggling to keep-up with exceeding memory requirements of modern applications. Adopted techniques such as multi layer cache hierarchy, out of order execution, are still falling short to mitigate the penalty incurred by memory accesses. Processing-in-memory (PIM) is emerging as a promising technique, which suggests to move memory-intensive kernels to memory for execution, instead of bringing the data to the processing unit. This technique has recently received a lot of traction among computer architecture researchers and increasing research activity investigating this technique indicates its potential to alleviate main memory performance bottlenecks to a great extent. In this paper, we characterize and identify memory-intensive HPC kernels, perform a first order evaluation of processing-in-memory technique for selected HPC kernels, quantify performance deviation and analyze the key factors that affect PIM efficiency.   